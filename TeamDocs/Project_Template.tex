\documentclass[12pt,]{article}
\usepackage{lmodern}
\usepackage{amssymb,amsmath}
\usepackage{ifxetex,ifluatex}
\usepackage{fixltx2e} % provides \textsubscript
\ifnum 0\ifxetex 1\fi\ifluatex 1\fi=0 % if pdftex
  \usepackage[T1]{fontenc}
  \usepackage[utf8]{inputenc}
\else % if luatex or xelatex
  \ifxetex
    \usepackage{mathspec}
  \else
    \usepackage{fontspec}
  \fi
  \defaultfontfeatures{Ligatures=TeX,Scale=MatchLowercase}
    \setmainfont[]{Times New Roman}
\fi
% use upquote if available, for straight quotes in verbatim environments
\IfFileExists{upquote.sty}{\usepackage{upquote}}{}
% use microtype if available
\IfFileExists{microtype.sty}{%
\usepackage{microtype}
\UseMicrotypeSet[protrusion]{basicmath} % disable protrusion for tt fonts
}{}
\usepackage[margin=2.54cm]{geometry}
\usepackage{hyperref}
\hypersetup{unicode=true,
            pdftitle={Examining the Hydrologic Properties of the Missouri River Basin},
            pdfauthor={Rachel Bash, Keqi He, Caroline Watson, and Haoyu Zhang},
            pdfborder={0 0 0},
            breaklinks=true}
\urlstyle{same}  % don't use monospace font for urls
\usepackage{graphicx,grffile}
\makeatletter
\def\maxwidth{\ifdim\Gin@nat@width>\linewidth\linewidth\else\Gin@nat@width\fi}
\def\maxheight{\ifdim\Gin@nat@height>\textheight\textheight\else\Gin@nat@height\fi}
\makeatother
% Scale images if necessary, so that they will not overflow the page
% margins by default, and it is still possible to overwrite the defaults
% using explicit options in \includegraphics[width, height, ...]{}
\setkeys{Gin}{width=\maxwidth,height=\maxheight,keepaspectratio}
\IfFileExists{parskip.sty}{%
\usepackage{parskip}
}{% else
\setlength{\parindent}{0pt}
\setlength{\parskip}{6pt plus 2pt minus 1pt}
}
\setlength{\emergencystretch}{3em}  % prevent overfull lines
\providecommand{\tightlist}{%
  \setlength{\itemsep}{0pt}\setlength{\parskip}{0pt}}
\setcounter{secnumdepth}{5}
% Redefines (sub)paragraphs to behave more like sections
\ifx\paragraph\undefined\else
\let\oldparagraph\paragraph
\renewcommand{\paragraph}[1]{\oldparagraph{#1}\mbox{}}
\fi
\ifx\subparagraph\undefined\else
\let\oldsubparagraph\subparagraph
\renewcommand{\subparagraph}[1]{\oldsubparagraph{#1}\mbox{}}
\fi

%%% Use protect on footnotes to avoid problems with footnotes in titles
\let\rmarkdownfootnote\footnote%
\def\footnote{\protect\rmarkdownfootnote}

%%% Change title format to be more compact
\usepackage{titling}

% Create subtitle command for use in maketitle
\providecommand{\subtitle}[1]{
  \posttitle{
    \begin{center}\large#1\end{center}
    }
}

\setlength{\droptitle}{-2em}

  \title{Examining the Hydrologic Properties of the Missouri River Basin}
    \pretitle{\vspace{\droptitle}\centering\huge}
  \posttitle{\par}
  \subtitle{\url{https://github.com/cwatson1013/Hydrologic_Data_Analysis_Final_Proj}}
  \author{Rachel Bash, Keqi He, Caroline Watson, and Haoyu Zhang}
    \preauthor{\centering\large\emph}
  \postauthor{\par}
    \date{}
    \predate{}\postdate{}
  
\usepackage{booktabs}
\usepackage{longtable}
\usepackage{array}
\usepackage{multirow}
\usepackage{wrapfig}
\usepackage{float}
\usepackage{colortbl}
\usepackage{pdflscape}
\usepackage{tabu}
\usepackage{threeparttable}
\usepackage{threeparttablex}
\usepackage[normalem]{ulem}
\usepackage{makecell}
\usepackage{xcolor}

\begin{document}
\maketitle
\begin{abstract}
The Missouri River provides critical water resources that drives the
region's agriculture, industry, and ecosystems. This is a region that
experiences surface water variability, characterized by damaging floods
and severe droughts, greatly impacting the agricultural production of
the area. This project highlights the changes in streamflow and water
quality over time, and identifies key characteristics of the
river\ldots{}.Twenty six sites across the lower Missouri River Basin
were examined in order to get a fuller picture of the Missouri River and
its tributaries over time.
\end{abstract}

\textless{}Information in these brackets are used for annotating the
RMarkdown file. They will not appear in the final version of the PDF
document\textgreater{}

\newpage
\tableofcontents 
\newpage
\listoftables 
\newpage
\listoffigures 
\newpage

\textless{}Note: set up autoreferencing for figures and tables in your
document\textgreater{}

\hypertarget{research-question-and-rationale}{%
\section{Research Question and
Rationale}\label{research-question-and-rationale}}

\textless{}Paragraph detailing the rationale for your analysis. What is
the significant application and/or interest in this topic? Connect to
environmental topic(s)/challenge(s).\textgreater{}

\textless{}Paragraph detailing your research question(s) and goals. What
do you want to find out? Include a sentence (or a few) on the dataset
you are using to answer this question - just enough to give your reader
an idea of where you are going with the analysis.\textgreater{}

\newpage

\hypertarget{dataset-information}{%
\section{Dataset Information}\label{dataset-information}}

\textless{}Information on how the dataset for this analysis were
collected, the data contained in the dataset, and any important pieces
of information that are relevant to your analyses. This section should
contain much of same information as the README file for the dataset but
formatted in a way that is more narrative.\textgreater{}

The data we are analyzing comes from the United States Geological Survey
(USGS) database called the National Water Information System interface,
or NWIS. We pulled data from the interface using the R package
\texttt{dataRetrieval}. Because we are interested in the lower Missouri
River basin, we pulled sites from each HUC4 subbasin from 1020 to 1030
(see Figure below). We chose these subbasins because they had a variety
of tributaries that all flowed into the Missouri River, and we wanted a
variety of river sizes and lengths. We filtered these subbasin queries
to only show us sites that had discharge, nitrogen, and phosphorus data.
Once we found the sites with all of this data, we chose 2 sites from
each HUC sub basin as our 22 ``best sites''. Our best sites had the
overall best time period range for all of our ``must have'' variables.

Only seven sites within our HUC subbasin boundary contained any high
frequency discharge and nitrogen data. Therefore, we also looked at
these 7 sites in order to do analyses and answer our research question
about flooding.

After doing initial data wrangling and analysis on our 22 ``best
sites'', we decided to pare it down further and only do subsequent
analyses on \textbf{10} sites. While we initially wanted to look at many
sites that were varied in size and location, we determined that it was
too many to look at and draw relevant conclusions from.

We have three main datasets:

\begin{verbatim}
1. The daily values dataset with our 22 "best sites"
2. The water quality dataset with our 22 "best sites", with only six sites that had total coliform data.
3. The high freqency dataset with 7 sites that contain both high freqency discharge and high frequency nitrogen data.
\end{verbatim}

\textless{}Add a table that summarizes your data structure. This table
can be made in markdown text or inserted as a \texttt{kable} function in
an R chunk. If the latter, do not include the code used to generate your
table.\textgreater{}

\textless{}C will do data table for water quality and daily values, R
will do for high freq\textgreater{}

\begin{landscape}\begin{table}[!h]

\caption{\label{tab:unnamed-chunk-2}Summary of Daily Value Data at 22 sites in the
      Missouri River Basin}
\centering
\resizebox{\linewidth}{!}{
\begin{tabular}{l|l|l|l|l|l|l|l|l|l|l|l|l|l|l|l|l|l|l|l|l|l|l|l|l|l|l|l|l|l|l|l|l|l|l|l|l|l|l}
\hline
  &       X & agency\_cd &    site\_no &         Date &   Discharge & Approval.Code &    parm\_cd &   sample\_tm & sample\_end\_dt & sample\_end\_tm & sample\_start\_time\_datum\_cd & tm\_datum\_rlbty\_cd &   coll\_ent\_cd & medium\_cd &     project\_cd & aqfr\_cd &  tu\_id & body\_part\_id &  hyd\_cond\_cd &  samp\_type\_cd &  hyd\_event\_cd &                              sample\_lab\_cm\_tx & remark\_cd &   result\_va &  val\_qual\_tx &    meth\_cd &  dqi\_cd &   rpt\_lev\_va &  rpt\_lev\_cd & lab\_std\_va & prep\_set\_no & prep\_dt &    anl\_set\_no &     anl\_dt &                                                                                                                               result\_lab\_cm\_tx &    anl\_ent\_cd &             startDateTime &            parameter\\
\hline
\rowcolor{gray!6}   & Min.   :     1 & USGS:658650 & Min.   :6609500 & 2009-10-20:    36 & Min.   :     0 & A      :618881 & Min.   : 60.00 & 11:00  :  1108 & Mode:logical & 14:15  :     8 & CDT :  8481 & K   :  5636 & USGS-WRD:  9272 & WS  : 13164 & NAWQA-SWS:   879 & Mode:logical & Mode:logical & Mode:logical & 9      :  5662 & 9      : 11792 & 9      : 11673 & tech samples                         :    90 & <   :   470 & Min.   :  0.0 & d      :   426 & ALGOR  :  4992 & A   :  6906 & Min.   :0.0 & DLDQC :   614 & Mode:logical & Mode:logical & Mode:logical & KJNT254A:    12 & Min.   :20010918 & The parameter 00665 was swapped from labcode 2333 to labcode 2759                                                                     :   160 & USGSNWQL:  2213 & 2018-05-08 10:30:00:     6 & Discharge       :645422\\
\hline
 & 1st Qu.:164663 & NA & 1st Qu.:6800000 & 2009-11-18:    36 & 1st Qu.:   154 & A e    : 21858 & 1st Qu.: 60.00 & 10:30  :   883 & NA's:658650 & 11:15  :     6 & CST :  4747 & T   :  7592 & USGS-NEL:  1051 & WSQ :    64 & 462900300:   689 & NA's:658650 & NA's:658650 & NA's:658650 & A      :  3373 & A      :   813 & Z      :   863 & tech sample                          :    75 & E   :    71 & 1st Qu.:  0.3 & oc     :   266 & KJ009  :  2579 & R   :  5574 & 1st Qu.:0.0 & LRL   :  1226 & NA's:658650 & NA's:658650 & NA's:658650 & KJNT135A:    11 & 1st Qu.:20070427 & The parameter 00665 was swapped from labcode 2333 to labcode 1984 because the result from labcode 2333 exceeded the calibration range.:    79 & NA's    :656437 & 1975-01-15 11:20:00:     4 & Total Nitrogen  :  6207\\
\hline
\rowcolor{gray!6}   & Median :329326 & NA & Median :6856600 & 1982-03-09:    35 & Median :   756 & P      :  4304 & Median : 60.00 & 10:00  :   831 & NA & 04:15  :     4 & NA's:645422 & NA's:645422 & USGSKSWC:   596 & NA's:645422 & 463106100:   649 & NA & NA & NA & 5      :  1397 & 7      :   515 & J      :   292 & tech sample;no sampling method given :    25 & NA's:658109 & Median :  0.7 & n      :    56 & CL084  :   701 & S   :   748 & Median :0.0 & LT-MDL:  1059 & NA & NA & NA & KJNT200A:    10 & Median :20120366 & The parameter 00665 was swapped from labcode 2333 to labcode 2759 because the result from labcode 2333 exceeded the calibration range.:    77 & NA & 1975-02-12 10:30:00:     4 & Total Phosphorus:  7021\\
\hline
 & Mean   :329326 & NA & Mean   :6828931 & 1980-08-12:    34 & Mean   : 10226 & P e    :   246 & Mean   : 71.54 & 12:00  :   750 & NA & 05:00  :     4 & NA & NA & USGS    :   517 & NA & 861100399:   542 & NA & NA & NA & 4      :  1235 & H      :    94 & 7      :   151 & tech samples;cross section from churn:    22 & NA & Mean   :  1.7 & doc    :    47 & CL021  :   609 & NA's:645422 & Mean   :0.0 & MRL   :    21 & NA & NA & NA & KJNT021A:     9 & Mean   :20111394 & Report level code updated Oct., Nov. 2015. Reference: NWQL TM 2015.02 (RLC: LT-MDL => DLDQC)                                          :    23 & NA & 1975-03-11 10:50:00:     4 & NA\\
\hline
\rowcolor{gray!6}   & 3rd Qu.:493988 & NA & 3rd Qu.:6892350 & 2017-07-25:    34 & 3rd Qu.:  4670 & P Ice  :    79 & 3rd Qu.: 60.00 & 11:30  :   747 & NA & 06:15  :     4 & NA & NA & USGSMOLS:   186 & NA & 463100300:   498 & NA & NA & NA & 8      :   692 & 5      :     8 & B      :   149 & BOTTLES OK                           :    18 & NA & 3rd Qu.:  2.2 & @d     :    11 & AKP01  :   411 & NA & 3rd Qu.:0.0 & NA's  :655730 & NA & NA & NA & KJNT023A:     9 & 3rd Qu.:20151101 & The holding time for the processing of this sample has been exceeded                                                                  :    12 & NA & 1975-07-22 10:00:00:     4 & NA\\
\hline
 & Max.   :658650 & NA & Max.   :6934500 & 1979-06-13:    33 & Max.   :739000 & (Other):    54 & Max.   :665.00 & (Other):  8903 & NA & (Other):    98 & NA & NA & (Other) :   188 & NA & (Other)  :  6154 & NA & NA & NA & (Other):   869 & (Other):     6 & (Other):   100 & (Other)                              :  3975 & NA & Max.   :100.0 & (Other):    62 & (Other):   392 & NA & Max.   :0.8 & NA & NA & NA & NA & (Other) :  2869 & Max.   :20191023 & (Other)                                                                                                                               :    28 & NA & (Other)            : 13200 & NA\\
\hline
\rowcolor{gray!6}   & NA & NA & NA & (Other)   :658442 & NA's   :13360 & NA's   : 13228 & NA & NA's   :645428 & NA & NA's   :658526 & NA & NA & NA's    :646840 & NA & NA's     :649239 & NA & NA & NA & NA's   :645422 & NA's   :645422 & NA's   :645422 & NA's                                 :654445 & NA & NA's   :645422 & NA's   :657782 & NA's   :648966 & NA & NA's   :655730 & NA & NA & NA & NA & NA's    :655730 & NA's   :655730 & NA's                                                                                                                                  :658271 & NA & NA's               :645428 & NA\\
\hline
\end{tabular}}
\end{table}
\end{landscape}

\begin{landscape}\begin{table}[!h]

\caption{\label{tab:unnamed-chunk-3}Summary of Water Quality Data in the
      Missouri River Basin}
\centering
\resizebox{\linewidth}{!}{
\begin{tabular}{l|l|l|l|l|l|l|l|l|l|l|l|l|l|l|l|l|l|l|l|l|l|l|l|l|l|l|l|l|l|l|l|l|l|l|l|l|l}
\hline
  &       X & agency\_cd &    site\_no &         Date & X\_00060\_00003 & X\_00060\_00003\_cd &    parm\_cd &   sample\_tm & sample\_end\_dt & sample\_end\_tm & sample\_start\_time\_datum\_cd & tm\_datum\_rlbty\_cd &   coll\_ent\_cd & medium\_cd &     project\_cd & aqfr\_cd &     tu\_id &  body\_part\_id &  hyd\_cond\_cd &  samp\_type\_cd &  hyd\_event\_cd &                              sample\_lab\_cm\_tx & remark\_cd &   result\_va &  val\_qual\_tx &    meth\_cd &  dqi\_cd &   rpt\_lev\_va &  rpt\_lev\_cd & lab\_std\_va & prep\_set\_no & prep\_dt &    anl\_set\_no &     anl\_dt &                                                                                                                               result\_lab\_cm\_tx &    anl\_ent\_cd &             startDateTime\\
\hline
\rowcolor{gray!6}   & Min.   :     1 & USGS:668800 & Min.   :6609500 & 1981-09-01:    56 & Min.   :     0 & A      :618881 & Min.   :   60.00 & 11:00  :  1819 & Mode:logical & 14:15  :    12 & CDT : 14622 & K   :  8908 & USGS-WRD: 15396 & BI  :    11 & NAWQA-SWS:  1411 & Mode:logical & Min.   :0 & Min.   :59.0 & 9      :  9726 & 9      : 20163 & 9      : 19933 & tech samples                         :   149 & <   :   472 & Min.   :     0.0 & d      :   426 & ALGOR  :  4994 & A   : 12661 & Min.   :0.0 & DLDQC :   616 & Mode:logical & Mode:logical & Mode:logical & KJNT254A:    12 & Min.   :20010918 & The parameter 00665 was swapped from labcode 2333 to labcode 2759                                                                     :   160 & USGS-WRD:  2206 & 2018-05-08 10:30:00:     9\\
\hline
 & 1st Qu.:167201 & NA & 1st Qu.:6800000 & 1981-09-02:    50 & 1st Qu.:   154 & A e    : 21858 & 1st Qu.:   60.00 & 10:30  :  1402 & NA's:668800 & 11:15  :    10 & CST :  8672 & T   : 14386 & USGS-NEL:  1690 & BP  :    41 & 462900300:  1089 & NA's:668800 & 1st Qu.:0 & 1st Qu.:59.0 & A      :  6963 & A      :  2127 & Z      :  2203 & tech sample                          :   127 & >   :     1 & 1st Qu.:     0.6 & oc     :   266 & KJ009  :  2581 & R   :  9500 & 1st Qu.:0.0 & LRL   :  1226 & NA's:668800 & NA's:668800 & NA's:668800 & KJNT135A:    11 & 1st Qu.:20070428 & The parameter 00665 was swapped from labcode 2333 to labcode 1984 because the result from labcode 2333 exceeded the calibration range.:    79 & USGSKSWC:   392 & 1979-06-13 11:00:00:     7\\
\hline
\rowcolor{gray!6}   & Median :334401 & NA & Median :6856600 & 2009-10-20:    43 & Median :   756 & P      :  4388 & Median :   60.00 & 10:00  :  1390 & NA & 05:00  :     7 & NA's:645506 & NA's:645506 & USGSKSWC:   988 & ON  :    26 & 463106100:   991 & NA & Median :0 & Median :94.0 & 5      :  2201 & 7      :   831 & J      :   530 & tech sample;no sampling method given :    50 & A   :    21 & Median :     3.8 & n      :    56 & PROBE  :  2013 & S   :  1133 & Median :0.0 & LT-MDL:  1059 & NA & NA & NA & KJNT200A:    10 & Median :20120404 & The parameter 00665 was swapped from labcode 2333 to labcode 2759 because the result from labcode 2333 exceeded the calibration range.:    77 & USGSNWQL:  2215 & 1975-01-15 11:20:00:     6\\
\hline
 & Mean   :334401 & NA & Mean   :6829150 & 2009-11-18:    43 & Mean   : 10227 & P e    :   246 & Mean   :   84.53 & 12:00  :  1277 & NA & 04:15  :     6 & NA & NA & USGS    :   805 & SB  :     8 & 463100300:   853 & NA & Mean   :0 & Mean   :81.2 & 4      :  1939 & H      :   149 & 7      :   248 & Billed FY19.                         :    48 & E   :   113 & Mean   :    88.9 & doc    :    47 & CL084  :   701 & NA's:645506 & Mean   :0.0 & MRL   :    21 & NA & NA & NA & KJNT021A:     9 & Mean   :20111448 & Report level code updated Oct., Nov. 2015. Reference: NWQL TM 2015.02 (RLC: LT-MDL => DLDQC)                                          :    23 & NA's    :663987 & 1975-02-12 10:30:00:     6\\
\hline
\rowcolor{gray!6}   & 3rd Qu.:501600 & NA & 3rd Qu.:6892350 & 1980-08-12:    42 & 3rd Qu.:  4670 & P Ice  :    79 & 3rd Qu.:   60.00 & 11:30  :  1232 & NA & 06:15  :     6 & NA & NA & USGSMOLS:   290 & WS  : 23094 & 861100399:   813 & NA & 3rd Qu.:0 & 3rd Qu.:94.0 & 8      :  1089 & 5      :    15 & B      :   224 & tech samples;cross section from churn:    33 & NA's:668193 & 3rd Qu.:     8.0 & @d     :    11 & EL003  :   678 & NA & 3rd Qu.:0.0 & NA's  :665878 & NA & NA & NA & KJNT023A:     9 & 3rd Qu.:20151102 & The holding time for the processing of this sample has been exceeded                                                                  :    12 & NA & 1975-03-11 10:50:00:     6\\
\hline
 & Max.   :668800 & NA & Max.   :6934500 & 1982-03-09:    42 & Max.   :739000 & (Other):    54 & Max.   :31501.00 & (Other): 15541 & NA & (Other):   150 & NA & NA & (Other) :   282 & WSQ :   114 & (Other)  : 10401 & NA & Max.   :0 & Max.   :94.0 & (Other):  1376 & (Other):     9 & (Other):   156 & (Other)                              :  6153 & NA & Max.   :400000.0 & (Other):    67 & (Other):  1566 & NA & Max.   :0.8 & NA & NA & NA & NA & (Other) :  2871 & Max.   :20191023 & (Other)                                                                                                                               :    29 & NA & (Other)            : 22627\\
\hline
\rowcolor{gray!6}   & NA & NA & NA & (Other)   :668524 & NA's   :23426 & NA's   : 23294 & NA & NA's   :646139 & NA & NA's   :668609 & NA & NA & NA's    :649349 & NA's:645506 & NA's     :653242 & NA & NA's   :668759 & NA's   :668759 & NA's   :645506 & NA's   :645506 & NA's   :645506 & NA's                                 :662240 & NA & NA's   :645506 & NA's   :667927 & NA's   :656267 & NA & NA's   :665878 & NA & NA & NA & NA & NA's    :665878 & NA's   :665878 & NA's                                                                                                                                  :668420 & NA & NA's               :646139\\
\hline
\end{tabular}}
\end{table}
\end{landscape}

\newpage

\hypertarget{exploratory-data-analysis-and-wrangling}{%
\section{Exploratory Data Analysis and
Wrangling}\label{exploratory-data-analysis-and-wrangling}}

\textless{}Include R chunks for 5+ lines of summary code (display code
and output), 3+ exploratory graphs (display graphs only), and any
wrangling you do to your dataset(s).\textgreater{}

\textless{}Include text sections to accompany these R chunks to explain
the reasoning behind your workflow, and the rationale for your
approach.\textgreater{}

\newpage

\hypertarget{analysis}{%
\section{Analysis}\label{analysis}}

\textless{}Include R chunks for 3+ statistical tests (display code and
output) and 3+ final visualization graphs (display graphs
only).\textgreater{}

\textless{}Include text sections to accompany these R chunks to explain
the reasoning behind your workflow, rationale for your approach, and the
justification of meeting or failing to meet assumptions of
tests.\textgreater{}

\newpage

\hypertarget{summary-and-conclusions}{%
\section{Summary and Conclusions}\label{summary-and-conclusions}}

\textless{}Summarize your major findings from your analyses. What
conclusions do you draw from your findings? Make sure to apply this to a
broader application for the research question you have
answered.\textgreater{}

\hypertarget{example-for-autoreferencing}{%
\subsection{Example for
autoreferencing}\label{example-for-autoreferencing}}

As seen by \autoref{fig:foo}, Absorbance values are not normally
distributed. This is expected, as we are dealing with ecological data.

\begin{figure}
\centering
\includegraphics{Project_Template_files/figure-latex/foo-1.pdf}
\caption{\label{fig:foo}Absorbance frequency}
\end{figure}


\end{document}
