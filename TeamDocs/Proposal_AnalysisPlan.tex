\documentclass[]{article}
\usepackage{lmodern}
\usepackage{amssymb,amsmath}
\usepackage{ifxetex,ifluatex}
\usepackage{fixltx2e} % provides \textsubscript
\ifnum 0\ifxetex 1\fi\ifluatex 1\fi=0 % if pdftex
  \usepackage[T1]{fontenc}
  \usepackage[utf8]{inputenc}
\else % if luatex or xelatex
  \ifxetex
    \usepackage{mathspec}
  \else
    \usepackage{fontspec}
  \fi
  \defaultfontfeatures{Ligatures=TeX,Scale=MatchLowercase}
\fi
% use upquote if available, for straight quotes in verbatim environments
\IfFileExists{upquote.sty}{\usepackage{upquote}}{}
% use microtype if available
\IfFileExists{microtype.sty}{%
\usepackage{microtype}
\UseMicrotypeSet[protrusion]{basicmath} % disable protrusion for tt fonts
}{}
\usepackage[margin=2.54cm]{geometry}
\usepackage{hyperref}
\hypersetup{unicode=true,
            pdftitle={Proposal and Analysis Plan},
            pdfauthor={Rachel Bash, Caroline Watson, Keqi He, Haoyu Zhang},
            pdfborder={0 0 0},
            breaklinks=true}
\urlstyle{same}  % don't use monospace font for urls
\usepackage{color}
\usepackage{fancyvrb}
\newcommand{\VerbBar}{|}
\newcommand{\VERB}{\Verb[commandchars=\\\{\}]}
\DefineVerbatimEnvironment{Highlighting}{Verbatim}{commandchars=\\\{\}}
% Add ',fontsize=\small' for more characters per line
\usepackage{framed}
\definecolor{shadecolor}{RGB}{248,248,248}
\newenvironment{Shaded}{\begin{snugshade}}{\end{snugshade}}
\newcommand{\AlertTok}[1]{\textcolor[rgb]{0.94,0.16,0.16}{#1}}
\newcommand{\AnnotationTok}[1]{\textcolor[rgb]{0.56,0.35,0.01}{\textbf{\textit{#1}}}}
\newcommand{\AttributeTok}[1]{\textcolor[rgb]{0.77,0.63,0.00}{#1}}
\newcommand{\BaseNTok}[1]{\textcolor[rgb]{0.00,0.00,0.81}{#1}}
\newcommand{\BuiltInTok}[1]{#1}
\newcommand{\CharTok}[1]{\textcolor[rgb]{0.31,0.60,0.02}{#1}}
\newcommand{\CommentTok}[1]{\textcolor[rgb]{0.56,0.35,0.01}{\textit{#1}}}
\newcommand{\CommentVarTok}[1]{\textcolor[rgb]{0.56,0.35,0.01}{\textbf{\textit{#1}}}}
\newcommand{\ConstantTok}[1]{\textcolor[rgb]{0.00,0.00,0.00}{#1}}
\newcommand{\ControlFlowTok}[1]{\textcolor[rgb]{0.13,0.29,0.53}{\textbf{#1}}}
\newcommand{\DataTypeTok}[1]{\textcolor[rgb]{0.13,0.29,0.53}{#1}}
\newcommand{\DecValTok}[1]{\textcolor[rgb]{0.00,0.00,0.81}{#1}}
\newcommand{\DocumentationTok}[1]{\textcolor[rgb]{0.56,0.35,0.01}{\textbf{\textit{#1}}}}
\newcommand{\ErrorTok}[1]{\textcolor[rgb]{0.64,0.00,0.00}{\textbf{#1}}}
\newcommand{\ExtensionTok}[1]{#1}
\newcommand{\FloatTok}[1]{\textcolor[rgb]{0.00,0.00,0.81}{#1}}
\newcommand{\FunctionTok}[1]{\textcolor[rgb]{0.00,0.00,0.00}{#1}}
\newcommand{\ImportTok}[1]{#1}
\newcommand{\InformationTok}[1]{\textcolor[rgb]{0.56,0.35,0.01}{\textbf{\textit{#1}}}}
\newcommand{\KeywordTok}[1]{\textcolor[rgb]{0.13,0.29,0.53}{\textbf{#1}}}
\newcommand{\NormalTok}[1]{#1}
\newcommand{\OperatorTok}[1]{\textcolor[rgb]{0.81,0.36,0.00}{\textbf{#1}}}
\newcommand{\OtherTok}[1]{\textcolor[rgb]{0.56,0.35,0.01}{#1}}
\newcommand{\PreprocessorTok}[1]{\textcolor[rgb]{0.56,0.35,0.01}{\textit{#1}}}
\newcommand{\RegionMarkerTok}[1]{#1}
\newcommand{\SpecialCharTok}[1]{\textcolor[rgb]{0.00,0.00,0.00}{#1}}
\newcommand{\SpecialStringTok}[1]{\textcolor[rgb]{0.31,0.60,0.02}{#1}}
\newcommand{\StringTok}[1]{\textcolor[rgb]{0.31,0.60,0.02}{#1}}
\newcommand{\VariableTok}[1]{\textcolor[rgb]{0.00,0.00,0.00}{#1}}
\newcommand{\VerbatimStringTok}[1]{\textcolor[rgb]{0.31,0.60,0.02}{#1}}
\newcommand{\WarningTok}[1]{\textcolor[rgb]{0.56,0.35,0.01}{\textbf{\textit{#1}}}}
\usepackage{longtable,booktabs}
\usepackage{graphicx,grffile}
\makeatletter
\def\maxwidth{\ifdim\Gin@nat@width>\linewidth\linewidth\else\Gin@nat@width\fi}
\def\maxheight{\ifdim\Gin@nat@height>\textheight\textheight\else\Gin@nat@height\fi}
\makeatother
% Scale images if necessary, so that they will not overflow the page
% margins by default, and it is still possible to overwrite the defaults
% using explicit options in \includegraphics[width, height, ...]{}
\setkeys{Gin}{width=\maxwidth,height=\maxheight,keepaspectratio}
\IfFileExists{parskip.sty}{%
\usepackage{parskip}
}{% else
\setlength{\parindent}{0pt}
\setlength{\parskip}{6pt plus 2pt minus 1pt}
}
\setlength{\emergencystretch}{3em}  % prevent overfull lines
\providecommand{\tightlist}{%
  \setlength{\itemsep}{0pt}\setlength{\parskip}{0pt}}
\setcounter{secnumdepth}{0}
% Redefines (sub)paragraphs to behave more like sections
\ifx\paragraph\undefined\else
\let\oldparagraph\paragraph
\renewcommand{\paragraph}[1]{\oldparagraph{#1}\mbox{}}
\fi
\ifx\subparagraph\undefined\else
\let\oldsubparagraph\subparagraph
\renewcommand{\subparagraph}[1]{\oldsubparagraph{#1}\mbox{}}
\fi

%%% Use protect on footnotes to avoid problems with footnotes in titles
\let\rmarkdownfootnote\footnote%
\def\footnote{\protect\rmarkdownfootnote}

%%% Change title format to be more compact
\usepackage{titling}

% Create subtitle command for use in maketitle
\providecommand{\subtitle}[1]{
  \posttitle{
    \begin{center}\large#1\end{center}
    }
}

\setlength{\droptitle}{-2em}

  \title{Proposal and Analysis Plan}
    \pretitle{\vspace{\droptitle}\centering\huge}
  \posttitle{\par}
    \author{Rachel Bash, Caroline Watson, Keqi He, Haoyu Zhang}
    \preauthor{\centering\large\emph}
  \postauthor{\par}
      \predate{\centering\large\emph}
  \postdate{\par}
    \date{October 25, 2019}


\begin{document}
\maketitle

\hypertarget{what-questions-will-your-team-address}{%
\subsection{What questions will your team
address?}\label{what-questions-will-your-team-address}}

\begin{enumerate}
\def\labelenumi{\arabic{enumi}.}
\tightlist
\item
  What is the current state of the rivers in the Southeastern Missouri
  River Basin and how has it changed over time?
\end{enumerate}

\begin{quote}
Be more specific. Go beyond water quality and quantity in this space.
How has discharge and nutrient levels changed over time. How can we
quantify the shifting extremes as it relates to water quality and
quantity. How do floods and droughts impact how nutrients get into
streams.
\end{quote}

\begin{quote}
How would land use affect the occurance of floods and droughts? What
other data can we pull in to address that? Pull in ESRI shapefile. dbf
but only if the shp file is in there at the same time. lulcc:
landuselandcoverchange
\end{quote}

\begin{enumerate}
\def\labelenumi{\arabic{enumi}.}
\setcounter{enumi}{1}
\item
  What effects did the March 2019 floods have on the water quality and
  quantity of rivers in the Missouri River Basin areas of interest?
\item
  What effects did the 2012-2013 drought have on the water quality and
  quantity of rivers in the Missouri River Basin areas of interest?
\item
  Given past and current data, what can we predict about the future
  state of water in the Missouri River Basin?
\end{enumerate}

\hypertarget{what-hypotheses-will-your-team-address}{%
\subsection{What hypotheses will your team
address?}\label{what-hypotheses-will-your-team-address}}

\begin{quote}
1a. Nutrients come from quickflow. 1b. Nutrients come from groundwater.
Give two options. If we see this phenomena, what are the mechanisms at
play?
\end{quote}

\begin{itemize}
\tightlist
\item
  Hypotheses should relate directly to your questions. Each numbered
  hypothesis should match up with the correponding numbered question.
\item
  There may be multiple working hypotheses for a single question. If
  this is the case, note each hypothesis as 1a, 1b. etc.
\end{itemize}

\begin{enumerate}
\def\labelenumi{\arabic{enumi}.}
\item
  Water quantity and quality in the Missouri River Basin has become more
  variable (more frequent storms and droughts) over time.
\item
  Nutrient concentrations and flow volume increased due to the floods of
  the 2019 flood.(we will create a hysteresis plot and a hydrograph for
  each site and nutrient of interest)
\end{enumerate}

\begin{enumerate}
\def\labelenumi{\alph{enumi}.}
\tightlist
\item
  Nitrate concentration will increase with flow. We hypothesize that
  hysteresis plots will exhibit a clockwise direction due to flushing of
  nutrients from the quickflow.
\end{enumerate}

\begin{enumerate}
\def\labelenumi{\arabic{enumi}.}
\setcounter{enumi}{2}
\item
  Discharge will decrease during a drought, also causing a decreased
  concentration of nitrogen due to less overland flow.
\item
  We predict that total flow in the Missouri River Basin is decreasing
  (non-stationary), and so the future situation of the river basin will
  see the continuation of current trends of decreasing overall volume of
  flow.
\end{enumerate}

\hypertarget{what-datasets-will-your-team-analyze}{%
\subsection{What dataset(s) will your team
analyze?}\label{what-datasets-will-your-team-analyze}}

USGS, Water Quality Portal

\begin{quote}
Run exact same line of code for the different HUC codes. Shooting for
number of sites or threshold for number of characterizations or samples
(ex: all sites with all variables were included, everything was
excluded. Second they have to be above x, etc. Methodologically
reproducible)
\end{quote}

\hypertarget{how-will-you-set-up-and-manage-your-project-repository}{%
\subsection{How will you set up and manage your project
repository?}\label{how-will-you-set-up-and-manage-your-project-repository}}

We have set up a git repository, so we will pull and push from the same
master.

\hypertarget{create-a-table-of-variables-you-will-analyze.}{%
\subsection{Create a table of variables you will
analyze.}\label{create-a-table-of-variables-you-will-analyze.}}

\begin{itemize}
\tightlist
\item
  Column 1: Variable
\item
  Column 2: Units (if known)
\item
  Column 3: Dependent (response) or independent (predictor) variable?
\item
  Column 4: To which hypothesis(es) does this variable pertain?
\end{itemize}

\emph{Note: You may not know all of the individual variables you plan to
analyze at this point. It is sufficient to describe what type of
variable you anticipate using (e.g., land cover) and decide on specifics
later}

\begin{Shaded}
\begin{Highlighting}[]
\KeywordTok{library}\NormalTok{(tidyverse)}

\NormalTok{Variable <-}\StringTok{ }\KeywordTok{c}\NormalTok{(}\StringTok{"discharge"}\NormalTok{, }\StringTok{"time"}\NormalTok{, }\StringTok{"nitrogen"}\NormalTok{, }\StringTok{"pH"}\NormalTok{, }\StringTok{"total coliform"}\NormalTok{, }
              \StringTok{"O2 Concentration"}\NormalTok{, }\StringTok{"Phosphorus"}\NormalTok{)}

\NormalTok{Units <-}\StringTok{ }\KeywordTok{c}\NormalTok{(}\StringTok{"cfs or m3/s"}\NormalTok{, }\StringTok{"UTC"}\NormalTok{, }\StringTok{"mg/L"}\NormalTok{, }\StringTok{"1"}\NormalTok{, }\StringTok{"cfu/100mL"}\NormalTok{, }\StringTok{"mg/L"}\NormalTok{, }\StringTok{"mg/L"}\NormalTok{)}

\NormalTok{TypeOfVariable <-}\StringTok{ }\KeywordTok{c}\NormalTok{(}\StringTok{"both"}\NormalTok{, }\StringTok{"independent"}\NormalTok{, }\StringTok{"dependent"}\NormalTok{, }\StringTok{"dependent"}\NormalTok{, }\StringTok{"dependent"}\NormalTok{, }
                    \StringTok{"dependent"}\NormalTok{, }\StringTok{"dependent"}\NormalTok{)}

\NormalTok{Hypothesis <-}\StringTok{ }\KeywordTok{c}\NormalTok{(}\StringTok{"all"}\NormalTok{, }\StringTok{"all"}\NormalTok{, }\StringTok{"2 and 3"}\NormalTok{, }\StringTok{"2 and 3"}\NormalTok{, }\StringTok{"2"}\NormalTok{, }\StringTok{"2 and 3"}\NormalTok{, }\StringTok{"2 and 3"}\NormalTok{)}

\NormalTok{Table <-}\StringTok{ }\KeywordTok{cbind}\NormalTok{(Variable, Units, TypeOfVariable, Hypothesis)}

\NormalTok{knitr}\OperatorTok{::}\KeywordTok{kable}\NormalTok{(Table)}
\end{Highlighting}
\end{Shaded}

\begin{longtable}[]{@{}llll@{}}
\toprule
Variable & Units & TypeOfVariable & Hypothesis\tabularnewline
\midrule
\endhead
discharge & cfs or m3/s & both & all\tabularnewline
time & UTC & independent & all\tabularnewline
nitrogen & mg/L & dependent & 2 and 3\tabularnewline
pH & 1 & dependent & 2 and 3\tabularnewline
total coliform & cfu/100mL & dependent & 2\tabularnewline
O2 Concentration & mg/L & dependent & 2 and 3\tabularnewline
Phosphorus & mg/L & dependent & 2 and 3\tabularnewline
\bottomrule
\end{longtable}

\hypertarget{what-tasks-will-your-team-conduct}{%
\subsection{What tasks will your team
conduct?}\label{what-tasks-will-your-team-conduct}}

\hypertarget{data-acquisition}{%
\subsubsection{Data acquisition}\label{data-acquisition}}

We will get our data from USGS (NWIS Site) and the Water Quality Portal
by looking for sites with good available data in the Missouri Region
(Hydrologic Region 10). We will pick 2-4 sites from each HUC 4 in the
1020-1029. We will choose sites that are both on the Missouri River and
sites that are on tributaries of the Missouri River. We will only pull
USGS sites on rivers.

We will clean our dataset by getting rid of unnecessary columns to make
analysis easier and will rename columns for clarity and conciseness.

We will also pull the watershed boundary dataset from USGS to map our
sites and water features.

\hypertarget{data-exploration}{%
\subsubsection{Data exploration}\label{data-exploration}}

Visualizing our site locations and basic discharge behavior will be our
first steps in exploring our data. We will visualize basic discharge
behavior and water quality data.

After exploration, we will address NA's if they are affecting our data.

\begin{quote}
How would we visualize; distributions across watersheds, plot over time;
violin plots, box plots, etc. to visualize range of data. Can be part of
final report to give context.
\end{quote}

\begin{quote}
can do C-Q plots (concentration quantity plots) without it being high
frequency data.
\end{quote}

\begin{quote}
smaller watersheds won't have as much data, but still worth it to
include to see picture at the smaller local scale.
\end{quote}

\hypertarget{data-wrangling}{%
\subsubsection{Data wrangling}\label{data-wrangling}}

We will create subset tables in order to focus on specific sites when we
address our flooding and drought questions. We will group our data by
year when we want to address trends over time. We will also wrangle data
by summarizing by month (and by day) from multiple years to get average,
maximum, and minimum values to compare. In order to find the recurrence
interval, we will need to wrangle data in order to find peak discharge.

\hypertarget{data-analysis-and-visualization}{%
\subsubsection{Data analysis and
visualization}\label{data-analysis-and-visualization}}

We will create hysteresis plots, recurrence interval plots, discharge
over time, typical discharge pattern over a year (plot with min, max,
and mean), nutrient concentration patterns over a year (specifically for
2019), pH over time and over flow, total cell count over time, total
coliform over time, plot hydrograph separation between baseflow and
quickflow. We will be creating dygraphs to visualize time series for
discharge and, if applicable, nutrient and oxygen concentrations.


\end{document}
